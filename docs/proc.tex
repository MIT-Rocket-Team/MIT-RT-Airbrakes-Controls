\documentclass{amsdtx}
\usepackage[margin=.75in]{geometry}
\usepackage{graphicx}
\usepackage{float}
\usepackage{eqnarray,amsmath}
\usepackage{amssymb}
\usepackage{url}
\usepackage[dvipsnames]{xcolor}
\usepackage{cancel}
\usepackage{multicol}
\usepackage[Glenn]{fncychap}
%\usepackage{hyperref}
\definecolor{b}{HTML}{3C78D8}
\definecolor{g}{HTML}{38761D}
\definecolor{r}{HTML}{AB0000}

\usepackage{listings}

\definecolor{codegreen}{HTML}{38761D}
\definecolor{codeblue}{HTML}{000076}
\definecolor{codegray}{rgb}{0.2,0.2,0.2}
\definecolor{codepurple}{HTML}{AB0000}
\definecolor{backcolour}{rgb}{0.95,0.95,0.95}

\lstdefinestyle{mystyle}{
    backgroundcolor=\color{white},   
    commentstyle=\color{codegreen},
    keywordstyle=\color{codepurple},
    numberstyle=\ttfamily\scriptsize\color{codeblue},
    stringstyle=\color{b},
    basicstyle=\ttfamily\footnotesize,
    breakatwhitespace=false,         
    breaklines=true,                 
    captionpos=b,                    
    keepspaces=true,                 
    numbers=left,                    
    numbersep=5pt,                  
    showspaces=false,                
    showstringspaces=false,
    showtabs=false,                  
    tabsize=4
}
\lstset{style=mystyle}

\newcommand{\bs}[1]{\boldsymbol{#1}}
\usepackage{bm}

\newcolumntype{M}[1]{>{\centering\arraybackslash}m{#1}}
\newcolumntype{N}{@{}m{0pt}@{}}

\newcommand{\vfs}[1]{\ensuremath \textbf{\textit{f}}\kern-0.1cm_{s_{#1}}}
\newcommand{\fs}[1]{\ensuremath f\kern-0.07cm_{s_{#1}}}
\newcommand{\F}{\ensuremath \textbf{F}\kern-0.07cm}
\newcommand{\cc}{\ensuremath\mu_{cc}}
\newcommand{\cg}{\ensuremath\mu_{cg}}
\usepackage{enumitem}

\usepackage[T2A,OT1]{fontenc}
\usepackage[utf8]{inputenc}
\usepackage[russian,english]{babel}

\title{\sc Airbrakes Control Algorithm}
\author{\sc Marc D Nichitiu, Akash Krishna | 8 January 2026}
\date{\sc MIT Rocket Team Simulations}
\begin{document}
\maketitle    
\section{Abstract}
We wish to choose a target apogee altitude using a variable-area airbrake system. Airbrakes are deployed using a servomotor that can expose a particular fraction of the Airbrake area to the freestream, thereby allowing the drag to depend on time. 

This algorithm seeks to enable a rocket having some default without-Airbrakes apogee altitude to lower this altitude to a close but particular height, to be chosen by the designer. Overall flow is as follows:
\begin{enumerate}
	\item A linear approximation (coefficient $k$) of the vertical velocity of the rocket as a function of time is characterized.
	\item A non-Airbrakes apogee altitude is determined according to this approximation.
	\item The difference between default apogee altitude and desired apogee altitude is calculated according to the pre-determined altitude choice. Let us denote it $\delta x$.
	\item A difference in the velocity decay slope, $\delta k$, is calculated based on $\delta x$.
	\item A difference in the drag $\dot x^2$ coefficient, $\delta B$, is calculated according to $\delta k$.
	\item The Airbrakes' deployed area $A(t)$ is calculated from $\delta B$. 
	\item Once Airbrakes are activated, they are to be extended according to $A(t)$, with a PIDF controller on position accuracy and acceleration.
\end{enumerate}

\section{Derivation: Perturbation Theory}
We observe from simulations and real launches that after engine burnout, velocity is roughly linear, following
\begin{align}
	\dot x(t) = -kt + v_0
\end{align}
where $k$ and $v_0$ are obtained from the rocket onboard sensors: a linear fit to several $\dot x(t)$ datapoints quickly obtains these.

This must satisfy the differential equation obtained from Newton's Second Law:
\begin{align}
	m\ddot x = \sum F_i = -mg -\frac{C_DA\rho \dot x^2}{2}
\end{align}
We write the coefficient of $\dot x^2$ as some $-mB$:
\begin{align}
	\ddot x = -g -B\dot x^2
\end{align}
We now substitute $\ddot x = -k$ and $\dot x = -kt+v_0$:
\begin{align}
	k = g + B(v_0-kt)^2
\end{align}
We wish to see how a perturbation $\delta k$ constrains a perturbation $\delta B$, removing all second/third order terms:
\begin{align}
	k+\delta k &= g + (B+\delta B)(v_0-(k+\delta k)t)^2 \\
	\Rightarrow \frac{k+\delta k-g}{B+\delta B} &= v_0^2-2(k+\delta k)tv_0+(k+\delta k)^2t^2 \\
	&= v_0^2 - 2ktv_0 - 2\delta kt v_0 + (k^2 + 2k\delta k + \cancel{\delta k^2})t^2 \\
	&= v_0^2 - 2ktv_0 - 2\delta kt v_0 + k^2t^2 + 2k\delta k t^2 \\
	\Rightarrow k+\delta k-g &= (B+\delta B)\left(v_0^2 - 2ktv_0 - 2\delta kt v_0 + k^2t^2 + 2k\delta k t^2\right) \\
	&= Bv_0^2 - 2Bktv_0 - 2B\delta kt v_0 + Bk^2t^2 + 2Bk\delta k t^2 \\
	&+ \delta Bv_0^2 - 2\delta Bktv_0 - \cancel{2\delta B\delta kt v_0} + {\delta B k^2t^2} + \cancel{2\delta Bk\delta k t^2}
\end{align}
We factor out $\delta B$:
\begin{align}
	\delta B\left(v_0^2 -2ktv_0+k^2t^2\right) + Bv_0^2 - 2Bktv_0 - 2B\delta ktv_0 + Bk^2t^2+2Bk\delta kt^2 = k+\delta k - g
\end{align}
Solving for $\delta B$ function of $\delta k$ gives
\begin{align}
	\delta B &= \frac{k+\delta k - g}{v_0^2 -2ktv_0+k^2t^2} - B\frac{v_0^2 - 2ktv_0  + k^2t^2 + 2k\delta kt^2- 2\delta ktv_0}{v_0^2 -2ktv_0+k^2t^2} \\
	&= \frac{k+\delta k - g}{v_0^2 -2ktv_0+k^2t^2} - B\left(1+2t\delta k\frac{kt- v_0}{(kt-v_0)^2}\right) \\
	&= \frac{k+\delta k - g}{(kt-v_0)^2} - B\left(1+\frac{2t\delta k}{kt-v_0}\right) \\
\end{align}
We had defined $B$ as 
\begin{align}
	B = \frac{C_DA_{\rm ref}\rho}{2m}
\end{align}
Thus, $\delta B$ is
\begin{align}
	\delta B = \frac{C_{D_{\rm airbrakes}}A(t)\rho}{2m}
\end{align}
Thus, the airbrakes area is 
\begin{align}
	A(t) = \frac{2m}{\rho C_{D_{\rm airbrakes}}}\delta B
\end{align}
The max apogee altitude is found from the linear velocity approximation: 
\begin{align}
	x_{\rm apog} = -\frac{k}{2}t_{\rm apog}^2 + v_0t_{\rm apog} + x_0 = \frac{v_0^2}{2k} + x_0
\end{align}
Thus, the perturbation is
\begin{align}
	\delta x = \frac{\partial}{\partial k}\left(\frac{v_0^2}{2k} + x_0\right)\delta k = -\delta k\frac{v_0^2}{2k^2}
\end{align}
which gives
\begin{align}
	\delta k = -\frac{2k^2}{v_0^2}\delta x
\end{align}
Thus, for a desired apogee altitude $x_d$, we have
\begin{align}
	\delta x \equiv x_d - x_{\rm apog} = x_d - {v_0^2}/{2k} + x_0
\end{align}
and so
\begin{align}
	A(t) &= \frac{2m}{\rho C_{D_{\rm airbrakes}}}\delta B  \\
	&= \frac{2m}{\rho C_{D_{\rm airbrakes}}}\left(\frac{k+\delta k - g}{(kt-v_0)^2} - B\left(1+\frac{2t\delta k}{kt-v_0}\right)\right) \\
	&= \frac{2m}{\rho C_{D_{\rm airbrakes}}}\left(\frac{k-\frac{2k^2}{v_0^2}\delta x - g}{(kt-v_0)^2} - B\left(1+\frac{2t\left(-\frac{2k^2}{v_0^2}\delta x\right)}{kt-v_0}\right)\right) \\
	&= \frac{2m}{\rho C_{D_{\rm airbrakes}}}\left(\frac{-{2k^2}\delta x +v_0^2(k- g)}{{v_0^2}(kt-v_0)^2} - B\left(1+\frac{-{4tk^2}\delta x}{{v_0^2}(kt-v_0)}\right)\right) \\
	\Rightarrow A(t) &= \frac{2m}{\rho C_{D_{\rm airbrakes}}}\left(\frac{-{2k^2}(x_d - {v_0^2}/{2k} + x_0) +v_0^2(k- g)}{{v_0^2}(kt-v_0)^2} - B\left(1+\frac{-{4tk^2}(x_d - {v_0^2}/{2k} + x_0)}{{v_0^2}(kt-v_0)}\right)\right)	
\end{align}
Thus, given a $k$, $v_0$ obtained from a linear fit to $\dot x(t)$, an $x_0$ from matching to a locally-averaged altitude point, a $B$ from the drag of the rocket without Airbrakes, and $\rho$ and $C_{D_{\rm airbrakes}}$ hardcoded, the $A(t)$ function can be fully obtained. The formula is not very computationally intensive.
\section{Implementation, Characterizations}


\end{document}













